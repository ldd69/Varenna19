\section{Data}
\label{sec:data}

\subsection{Data distributions}
\label{eq:data-distr}
For each experiment we want to check the experimental covariance matrix that has
been implemented in the \nnpdf\ code. Ideally we want this check to be
implemented as part of \valid.

We should always work in the basis of eigenvectors of the covariance matrix. The
first step when implemnting experiments should be the diagonalization of the
covariance matrix, checking the eigenvectors. The matrix should also be
projected on the nearest positive definite matrix, so that any numerical
instability is under control.

The basis of eigenvectors of the covariance matrix is a natural way to
represent the data before performing and training, validation or test splits
since the new data points are uncorrelated form the perspective from the fit.
Even if we are not transforming the data we should be concerned when the condition
number of a given dataset covariance matrix, or given experiment covariance matrix
is high. In this section we being by simply listing the condition numbers of
the global datasets and experiments.

If the condition numbers are found to be high then we can try a number of prescriptions, either
normalising by the data central value or setting any eigenvalues below some threshold according
to an acceptable condition number and the maximum eigenvalue.

Will update this section after Zahari's talk/speaking with Zahari.
